



%%%%%%%%%%%%%%%%%%%%%%%%%%%%%%%%%%%%%%%%%
%ONLY DIFFERENCE FROM HDR IS NO DOC CLASS INCLUDED%
%%%%%%%%%%%%%%%%%%%%%%%%%%%%%%%%%%%%%%%%%








%  paper draft, symmetric hypergraphs, summer 2015 and summer 2016
%% revision 4/2017 for submission to JPhysA
%%%% this revision is for minor edits suggested by referees in 4/7/2017 report

%% draft style for writing process



%%%remove [titlepage] here for flush title
%\documentclass[titlepage]{article}
%\documentclass{revtex4-1}


%%% revtex aip style for journal of mathematical physics
%%% move \author and \title declarations before/after ``\begin{document}''
%\documentclass[aip]{revtex4-1}

\def\changemargin#1#2{\list{}{\rightmargin#2\leftmargin#1}\item[]}
\let\endchangemargin=\endlist 





%%%%%%%%%%%PythonTex%%%%%%%%%%%%%%%%%%

% Engine-specific settings
% Detect pdftex/xetex/luatex, and load appropriate font packages.
% This is inspired by the approach in the iftex package.
% pdftex:

%%%%%%%%%%%%%%%%%%%%%%%%%
%if trouble remove % from next rows before % xetex:
%%%%%%%%%%%%%%%%%%%%%%%%%%%
   %\ifx\pdfmatch\undefined
   %\else
       %\usepackage[T1]{fontenc}
       %\usepackage[utf8]{inputenc}
   %\fi
% xetex:
%\ifx\XeTeXinterchartoks\undefined
%\else
    %\usepackage{fontspec}
    %\defaultfontfeatures{Ligatures=TeX}
%\fi
% luatex:
%\ifx\directlua\undefined
%\else
    %\usepackage{fontspec}
%\fi
% End engine-specific settings

\usepackage{amsmath,amssymb}
%\usepackage{fullpage}
\usepackage{graphicx}
\usepackage[svgnames]{xcolor}
\usepackage{url}
\urlstyle{same}

\usepackage[makestderr]{pythontex}
\restartpythontexsession{\thesection}


\usepackage[framemethod=TikZ]{mdframed}

\newcommand{\pytex}{Python\TeX}

%%%%quick build setup user setting
%%  here->    pdflatex -synctex=1 -interaction=nonstopmode %.tex|pythontex %.pytexcode|pdflatex -synctex=1 -interaction=nonstopmode %.tex|"C:/Program Files/Adobe/Reader 11.0/Reader/AcroRd32.exe" %.pdf

%%%%%%%%%%%PythonTexEnd%%%%%%%%%%%%%%%%%%%%%%%%%%%%%%%%%%%%%%%%%%%%%%%%%%%%%%%%%%%%%%%%%%%%%%%%%%%%%%%%%%%%%%%%%%%%%%%%%%%%%%%%%%%%%%%%%%%%%%%%%%%%%%%%%%%%%%%%%%%%%%%%%%%%%%%%%%%%%%%%%%%%%%%%%%%%%



%%% NOTE for J Math Phys submission 9/2016, had to comment out the next
%%% line to get the paper to build correctly on their server
%\pdfoutput=1
\usepackage{amssymb}
\usepackage{epsfig}
\usepackage{color}
\usepackage{amsfonts}
\usepackage{amsmath}
\usepackage{bbold}
\usepackage{graphicx}
%\usepackage{pdfpages}
\usepackage{mathtools}
\usepackage{tikz}
\usepackage{relsize}
\usepackage{amsmath}
\usepackage{kbordermatrix}

\usepackage{graphicx}
\usepackage{anyfontsize}
\usepackage{color}
\usepackage{epsfig}
\usepackage{amssymb}
\usepackage{amsfonts}
\usepackage{amsmath}
%\usepackage{xypic}
\usepackage{tikz}
\usepackage{mathtools}
\usepackage{array}
%\usepackage{a0size}


\usepackage{tikz}
\usepackage{circuitikz}

\usepackage{pgf,tikz,pgfplots}
\pgfplotsset{compat=1.15}
\usepackage{mathrsfs}
\usetikzlibrary{arrows}

\usepackage{mdframed}

\usepackage{booktabs}
\usepackage{chngpage}
\usepackage{xcolor}
\usepackage{colortbl}

\usepackage{dashrule}

\usepackage{todonotes}

\usepackage{hyperref}

\usepackage{sagetex}

%\usepackage{xypic}
%\input xy


%%%%Blue Hyperlinks
\hypersetup{colorlinks=true,linkcolor=blue, linktocpage}


%\addtolength{\oddsidemargin}{.05in}
%\addtolength{\topmargin}{-.50in}
%\addtolength{\textheight}{1in}
%\reversemarginpar
%\addtolength{\headsep}{.25in}
 

%\parskip = 5pt plus 0pt minus 0pt

% invisible text ``spacer''
\newcommand{\spacer}{\rule[0cm]{0cm}{0cm}}

%%%%%%%%%%%%%%%% theorems, remarks and proofs %%%%%%%%%%%%%%%%%%%%%%

\newtheorem{theorem}{Theorem}
\newtheorem{lemma}{Lemma}
\newtheorem{proposition}{Proposition}

\newcounter{statementnumber}
\renewcommand{\thestatementnumber}{\arabic{section}.\arabic{statementnumber}}

%thm environment
\newenvironment{thm}[2]{\refstepcounter{statementnumber} \label{#2}
%\par \noindent {\bf #1~\thestatementnumber.}  
\par \noindent {\bf (\thestatementnumber) #1.}
\begin{em}}{\end{em} \par}

%rem environment
\newenvironment{rem}[2]{\refstepcounter{statementnumber} \label{#2}
\par \noindent {\bf (\thestatementnumber) #1.}
}{\par}

%\newcommand{\proofend}{$\rule{2mm}{2mm}$}

\newcommand{\proofend}{\nopagebreak \hfill
{\framebox{\rule{.5ex}{0ex}\rule{0ex}{.5ex}}} \par}

\newenvironment{proof}{\par\noindent\textsc{Proof.}}{\nopagebreak\spacer\hfill $\square$}




%%%%%%%%%%%%%%%%%% end of thm, rem, proof set up %%%%%%%%%%%%%%%%%%

%%%%%%%%%%%%%%%%%%% placement of figures in the margin %%%%%%%%%%%%%%%%%%%%%%%%%
\newcounter{figurecount}
%\renewcommand{\thefigurecount}{\thesection .\arabic{figurecount}}

% command for figures in the margin
% arguments: figure input file name, reference label, figure horizontal
% offset, figure vertical offset, ``Figure'' label horizontal offset,
% ``Figure'' label vertical offset, caption


%%%%%Cell coloring for booktabs
\makeatletter
\newcommand{\ccell}[3][]{%
  \kern-\fboxsep
  \if\relax\detokenize{#1}\relax
    \expandafter\@firstoftwo
  \else
    \expandafter\@secondoftwo
  \fi
  {\colorbox{#2}}%
  {\colorbox[#1]{#2}}%
  {#3}\kern-\fboxsep
}
\makeatother



%%%%%%5

\newcommand{\marginfig}[7]{
\marginpar{
\refstepcounter{figurecount}      % increment margin figure counter
\label{#2}                        % label the figure
\vspace*{#4}                      % vertical adjustment for figure
\spacer\\
\hspace*{#3}                      % horizontal adjustment for figure
\input{#1}\\
\vspace*{#6}                      % vertical adjustment for ``Figure'' label
\spacer\\
\hspace*{#5}                      % horizontal adjustment for ``Figure'' label
\shortstack{
Figure~\ref{#2}\\ #7
}
\spacer}}
%%%%%%%%%%%%%%%%%%%%%%%%%%%%%%%%%%%%%%%%%%%%%%%
\newcommand\scalemath[2]{\scalebox{#1}{\mbox{\ensuremath{\displaystyle #2}}}}


%%%%%%%%%%%%%%%%%%%%%%%%%%%%%%%%%

% capital D derivative symbol
\def\D{\mbox{D}}
% space for dx in integrands
\def\d{\; d}
% -1 exponent for inverse
\def\inv{^{-1}}
% composition of functions circle
% \def\of{\, \mbox{\small{$\circ$}} \:}
\def\of{\mathbin{\circ}}
% shortcut for displaystyle
\def\dsp{\displaystyle}
% shortcut for displaystyle in math mode
\def\mb{$\begin{displaystyle}}
\def\me{\end{displaystyle}$\ }
\def\id{\mbox{\rm id}}
\def\Id{\mbox{\rm Id}}
\def\wt{\mbox{\rm wt}}
\def\glb{\mbox{\rm glb}}
\DeclareMathOperator{\rank}{rank}
\DeclareMathOperator{\tr}{tr}
\DeclareMathOperator{\Aut}{Aut}
\DeclareMathOperator{\End}{End}
\newcommand{\upk}{^{(k)}}
\newcommand{\downk}{_{(k)}}
\newcommand{\ket}[1]{\left| #1 \right\rangle}
\newcommand{\bra}[1]{\left\langle #1 \right|}
\newcommand{\inprod}[2]{\left\langle #1 | #2 \right\rangle}
\newcommand{\abs}[1]{\left| #1 \right|}
\newcommand{\twotwo}[4]{\left[ \begin{array}{cc} #1 & #2 \\
                        #3 & #4 \end{array} \right]}
\DeclareMathOperator{\noncross}{NC} % noncrossing partitions
\DeclareMathOperator{\noncrossadj}{NCAT} % noncrossing partitions with adjacent
                                % transpositions (only)
\DeclareMathOperator{\re}{Re}
\DeclareMathOperator{\Ad}{Ad}
\DeclareMathOperator{\ad}{ad}
\newcommand{\real}{\mbox{\rm Re}}
\newcommand{\imag}{\mbox{\rm Im}}
\newcommand{\Stab}{\mbox{\rm Stab}}
% notation for multi-index operations
\DeclareMathOperator{\ND}{ND}
\DeclareMathOperator{\NA}{NA}
\DeclareMathOperator{\None}{N1}
\DeclareMathOperator{\B01}{B}

%bold face caps
\newcommand{\R}{{\mathbb R}}
\newcommand{\Aff}{{\mathbb A}}
\newcommand{\Proj}{{\mathbb P}}
\newcommand{\C}{{\mathbb C}}
\newcommand{\Quat}{{\mathbb H}}
\newcommand{\F}{{\mathbb F}}
\newcommand{\A}{{\mathbb A}}
\newcommand{\Q}{{\mathbb Q}}
\newcommand{\Z}{{\mathbb Z}}
\newcommand{\N}{{\mathbb N}}

\newcommand{\toppauli}{\mbox{\small \rm top}}
\newcommand{\psiG}{\ket{\psi_{G}}}
\newcommand{\mathfrik}{ }
\newcommand{\1}{{\mathbb 1}}

\newcommand{\sex}{\textbf{Symbolic Ex.} }
\newcommand{\eex}{\textbf{Explicit Ex.} }

%bold face italics for definitions
\font\bfit=cmbxti10




%\title{\textbf{Classification of Set Structures:} A Novice Attempt}

%\author{Alexander J. Heilman}

%    \pagestyle{myheadings}
%\markright{Set Structures}
 %  \setlength{\parindent}{0in}

%\date{revised: 30 July 2018}

%% \author{David W. Lyons}
%% \author{Nathaniel P. Gibbons}
%% \author{Mark A. Peters}
%% \author{Daniel J. Upchurch}
%% \author{Scott N. Walck}
%% \author{Ezekiel W. Wertz}

% arXiv version 1 submitted 9/5/2016
% j math phys submission 9/22/2016, minor typos fixed since arXiv v1






%%% \title and \author *after* ``\begin{document}" for revtex aip style
%\title{Local unitary stabilizers of symmetric hypergraph states}
%\author{David W. Lyons, Nathaniel P. Gibbons, Mark A. Peters, and Scott N. Walck}


%%% check format and placement of \maketitle and abstract for revtex aip j. math. phys. submission
%%% \title and \author *before* ``\begin{document}" for article style draft